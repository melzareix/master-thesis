% !TeX root = ../main.tex

\chapter{Introduction}\label{chapter:introduction}

\section{Motivation}
String processing tasks are frequently used in analytical queries that power business intelligence. Aside from sub-string matching, which is given by the \texttt{\textbf{like}} operator in SQL, popular DBMSs also support regular expressions.

Regular expressions are textual patterns used for text analytics and information extraction using standardized concise syntax. Regular expressions have a more complex usage of wildcards and can be constructed to create a more complex and precise match than the \texttt{\textbf{like}} operator. 

Regular Expressions have been used in many applications throughout the data analytics pipeline in recent years. Such applications include recommender systems \cite{recsys1}, natural language processing (NLP) \cite{nlp1, nlp2}, and graph processing \cite{graph1}. These applications typically involve data sets of an increasingly large size, reaching tens or hundreds of Gigabytes. The large datasets raise a challenge for regular expressions engines in DBMSs regarding executing the regular expressions queries efficiently.

Traditional general-purpose regular expressions engines (e.g., Boost Regex \cite{Boost}, RE2 \cite{re2}) are interpreted engines that, for each regular expression, build an in-memory data structure (or a small program) that can match that regular expression. This approach works well for small datasets, but larger datasets with millions of rows suffer from (1) Indirection via memory access which wastes 100s of CPU cycles \cite{cpumemgap}. (2) Data is not kept in CPU registers, and contents are evicted regularly.

The LLVM compiler infrastructure \cite{llvm} consists of reusable modular compiler tools that provide just-in-time (JIT) compilation and optimizations. \citet{querycomp} introduced a unique compilation method that uses the LLVM compiler infrastructure to convert a database query into compact and efficient machine code. Inspired by his work, this thesis investigates the compilation of regular expressions match queries into an intermediate representation that can be compiled to efficient native machine code. We assume that doing this should be more performant than other interpreted engines when used against large data sets, and a large number of runs amortize the cost of compilation.

\section{Research Question}\label{researchq}

The main research question of this thesis is: Can a code-generating regular expressions engine built on top of LLVM be performant?

To answer this question, we built a prototype regular expressions engine with Unicode support and dynamic compilation built on top of the LLVM compiler infrastructure. The engine takes as an input a regular expression and a text file containing the text to match. As a first step, it generates a data structure that matches the pattern. It then passes the data structure to a code-generation module (built on top of LLVM) responsible for compiling it into efficient machine code. Finally, The generated code processes the input text and does the regular expressions matching.

We also compare our implementation with other popular regular expressions engines to evaluate the performance and viability of our idea. 